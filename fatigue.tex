\documentclass[preprint,5p,twocolumn,11pt,sort&compress]{elsarticle}

\usepackage{amssymb}
\usepackage{amsthm}
\setlength{\mathindent}{0pt}
\usepackage{tabularx}
\usepackage{graphicx}
\usepackage{epsfig}
\usepackage{textcomp}
\usepackage{subfigure}
\usepackage{natbib}
\usepackage[colorlinks,linkcolor=red,anchorcolor=blue,citecolor=blue]{hyperref}
\usepackage{color,soul}
\usepackage{multirow}
\usepackage{overpic}
% \usepackage{ctex}
\usepackage{caption}

\newcommand{\bfsigma}{{\mbox{\boldmath{$\sigma$}}}}
\newcommand{\bfepsilon}{{\mbox{\boldmath{$\varepsilon$}}}}
\newcommand{\dotbfepsilon}{{\mbox{\boldmath{$\dot\varepsilon$}}}}
\newcommand{\dotbfsigma}{{\mbox{\boldmath{$\dot\sigma$}}}}
\newcommand{\bftau}{{\mbox{\boldmath{$\tau$}}}}
\newcommand{\bfpsi}{{\mbox{\boldmath{$\psi$}}}}
\newcommand{\bfphi}{{\mbox{\boldmath{$\phi$}}}}
\newcommand{\bfalpha}{{\mbox{\boldmath{$\alpha$}}}}
\newcommand{\bfbeta}{{\mbox{\boldmath{$\beta$}}}}
\newcommand{\bfK}{{\bf K}}
\newcommand{\bff}{{\bf f}}
\newcommand{\bfn}{{\bf n}}
\newcommand{\bfm}{{\bf m}}
\newcommand{\bft}{{\bf t}}
\newcommand{\bfu}{{\bf u}}
\newcommand{\bfw}{{\bf w}}
\newcommand{\bfa}{{\bf a}}
\newcommand{\bfb}{{\bf b}}
\newcommand{\bfs}{{\bf s}}
\newcommand{\bfB}{{\bf B}}
\newcommand{\bfD}{{\bf D}}
\newcommand{\bfnabla}{{\mbox{\boldmath{$\nabla$}}}}
\newcommand{\bfDelta}{{\mbox{\boldmath{$\Delta$}}}}
\newcommand{\bfkappa}{{\mbox{\boldmath{$\kappa$}}}}
\newcommand{\bfN}{{\bf N}}
\newcommand{\bfT}{{\bf T}}
\newcommand{\bfG}{{\bf G}}
\newcommand{\bfH}{{\bf H}}
\newcommand{\dd}{{\rm d}}
\newcommand{\marked}[1]{\textcolor{red}{#1}}

%\bibliographystyle{elsarticle-num}
\graphicspath{{F:/Cloud/GitHub/fatigue/figs/}{F:/Cloud/GitHub/doctor/figs/python/}{F:/Cloud/GitHub/doctor/figs/ppt/}{F:/Cloud/GitHub/doctor/figs/sem/}}

\journal{International Journal of Fatigue}

\begin{document}

\captionsetup[figure]{labelfont={bf},font={footnotesize},name={Fig.},labelsep=period}

\captionsetup[table]{labelfont={bf},font={footnotesize},name={Tabel},labelsep=period}

\sethlcolor{yellow}

\begin{frontmatter}

%% Title, authors and addresses

%% use the tnoteref command within \title for footnotes;
%% use the tnotetext command for theassociated footnote;
%% use the fnref command within \author or \address for footnotes;
%% use the fntext command for theassociated footnote;
%% use the corref command within \author for corresponding author footnotes;
%% use the cortext command for theassociated footnote;
%% use the ead command for the email address,
%% and the form \ead[url] for the home page:
%% \title{Title\tnoteref{label1}}
%% \tnotetext[label1]{}
%% \author{Name\corref{cor1}\fnref{label2}}
%% \ead{email address}
%% \ead[url]{home page}
%% \fntext[label2]{}
%% \cortext[cor1]{}
%% \address{Address\fnref{label3}}
%% \fntext[label3]{}

\title{Life Assessment of Multi-Axial Thermomechanical Fatigue of A Nickel-Based Superalloy}

%% use optional labels to link authors explicitly to addresses:
%% \author[label1,label2]{}
%% \address[label1]{}
%% \address[label2]{}

\author{Jingyu SUN\fnref{label1}}
\author{Huang YUAN\corref{cor1}\fnref{label1}}

%\address[authorlabel1]{Department of Mechanical Engineering, University of Wuppertal, Germany}
\address[label1]{School of Aerospace Engineering, Tsinghua University, Beijing, China\fnref{label1}}
\address[label2]{Department of Civil Engineering, Technical University of Darmstadt, Germany\fnref{label2}}
\cortext[cor1]{Corresponding author.}
\ead{yuan.huang@tsinghua.edu.cn}

\begin{abstract}

Turbine components are generally under mechanical and thermal loads. Recent works confirm significant effects of thermomechanical loads to fatigue life assessment. In the present work, extensive experiments are performed for a nickel-based superalloy under both iso-thermal and thermo-mechanical loading conditions to quantify influence of thermal phase angle and loading non-proportionality. Based on experimental data a thermomechanical loading parameter is introduced to assess fatigue failure. The thermomechanical fatigue life can be calibrated based on the present concept reasonably.
\end{abstract}

%\include{debut}
\begin{keyword}
% keywords here, in the form: keyword \sep keyword
Thermomechanical Fatigue (TFM) \sep Multi-Axial Fatigue \sep Fatigue Damage \sep Fatigue Life \sep Nickel-based super alloy

% PACS codes here, in the form: \PACS code \sep code
% \PACS
\end{keyword}
\end{frontmatter}

% main text
\section{Introduction}
Gas turbine components experience severe cyclic multi-axial mechanical loadings and thermal loadings. The increasing operating temperature in gas turbines is pushing materials closer to their operating limits. Quantifying mechanical behavior and fatigue performance of the components under the more realistic conditions becomes necessary \cite{Harrison1996}.

In the past decades the high temperature superalloy In718 was extensively tested under isothermal loading conditions, especially by different aero engine manufacturers. Engineering design is based on uniaxial data, so that the influence of the loading multi-axiality was not clarified. In literature very few systematic experimental data can be found. For engineering applications numerous investigations on the isothermal low cycle fatigue of the nickel-based superalloy were performed \cite{Mahobia2014, Chen2016, William1995}\marked{(Add two handbooks here)}. Additional series of strain controlled fatigue tests at $649^\circ$C was been completed by Kim \cite{kim1988elevated} and Nelson \cite{nelson1992creep} from NASA with strain rate from $2\times10^{-4}$s$^{-1}$ to $5\times10^{-3}$s$^{-1}$.
 \marked{(Overview on isothermal fatigue is too short. More publications on IN718, incl. Socie \& Co. should be mentioned. Our MTU report contains data from them in 1989)}. Life design of the turbine components is based on isothermal results, although the real loading history in turbine is essentially under varying temperature. One believes that the material fatigue under higher temperature is more critical than that under thermomechanical condition with both varying mechanical and thermal loads. From engineering application view point, fatigue life prediction of isothermal components is more feasible than the thermomechanical life design. However, experiments reveal that the varying temperature changes fatigue damage mechanisms and may accelerate material failure significantly. Recently quantifying thermomechanical fatigue damage becomes an important design issue, especially for high performance turbines.

Thermomechanical fatigue (TMF) means fatigue under both cyclic mechanical loads as well as cyclic temperature. The study of Sehitoglu \marked{(In this paragraph more literature and results on TMF are necessary.)} showed that material experiences different damage processes under TMF and fatigue life is sensitive to the loading configuration. In past years many TMF tests were published mainly for uniaxial loading cases \cite{Evans2008, Kulawinski2015, Remy2003, Bauer2009}. The major issue was to demonstrate effects of the loading phase angles and to correlate fatigue life with the phase angle. A phenomenological fatigue models was proposed  in \cite{Vose2013} to predict the material fatigue life under isothermal and thermo-mechanical loading conditions. However, these works are limited to the uniaxial loading condition. It is well known the deformation and damage mechanisms under multiaxial loads can significantly differ from those under uniaxial loading \cite{Fang2015, Kang2004, Chen2004}. Most components in turbine engines typically experience significant variations in multiaxial states of stress, strain and temperatures under non-isothermal conditions. There are few works on the multiaxial TMF fatigue life published \cite{Brookes2010}.

The present paper considers thermomechanical fatigue under tension-torsion loading conditions and discusses multiaxial TMF fatigue. Both elastic-plastic behavior and fatigue life under multiaxial TMF loading condition are investigated. It is shown that the TMF affects fatigue performance of the material significantly.

\section{Experiments}
\subsection{Material specification}
% Inconel 718 is one of the most popular nickel-based superalloys used in turbine engines. The material possesses optimal property at both room temperature as well as at elevated temperature and does not reveal obvious creep up to $650^\circ$C. 
The investigated material is the nickel-base superalloy Inconel 718, which was provided by ThyssenKrupp VDM GmbH.
in the form of rods of 20mm diameter
The rods were solution-treated at 980$^{\circ}$C for one and a half hour then cooled to room temperature by water.
Then they were aged for eight hours at 720$^{\circ}$C, furnace cooled at 56$^{\circ}$C/h to 621$^{\circ}$C, where they were held for eight hours and forced air cooling to room temperature.
The chemical composition of the nickel-base superalloy Inconel 718 in the research is given in Table \ref{Tab:ChemicalCompositionofIN718}.

The initial microstructure of the superalloy Inconel 718 used in this study is shown by Fig. \ref{Fig:microstructure_200X}.
After polishing, the sample was electrolytically etched with oxalic acid.
According to the linear intercept method, the average diameter of the initial grains is about 20 $\mu$m.

\begin{table*}[htbp]
  \centering
  \caption{Chemical composition of Inconel 718 (wt. \%) in the present work}\vspace{0.1cm}
    \begin{tabular}{llllllllll}
    \hline
    C     & S     & Cr    & Ni    & Mn    & Si    & Mo    & Ti    & Nb    & Cu \\
    \hline
    0.02  & $<$0.001 & 18.53 & 53.44 & 0.05  & 0.06  & 3.06  & 0.99  & 5.30  & 0.04 \\
    \hline
    Fe    & P     & Al    & Pb    & Co    & B     & Ta    & Se    & Bi    &  \\
    \hline
    17.71 & 0.007 & 0.56  & 0.0002 & 0.13  & 0.004 & $<$0.01 & $<$0.0003 & $<$0.00003 &  \\
    \hline
    \end{tabular}%
  \label{Tab:ChemicalCompositionofIN718}%
\end{table*}%

\begin{figure}[!htp]
\centering
\includegraphics[width=8.5cm]{microstructure_200X.png}
\caption{Initial microstructure of the specimen section.}
\label{Fig:microstructure_200X}
\end{figure}

\begin{figure}[!htp]
\centering{\includegraphics[width=8.5cm]{IN718_Multiaxial_Specimen.pdf}}
\caption{Geometry of the specimens investigated in the present work.}
\label{Fig:Specimen}
\end{figure}

\subsection{Specimens}
The thin-walled tubular specimen, as shown in Fig. \ref{Fig:Specimen}, was used in the present work. The specimen dimension is in accordance with ASTM E2207 \cite{ASTM2014} and matches the extensometer, the furnace and the induction coil. The experiments confirm that the specimen geometry influences the temperature distribution. Consequently the final geometry is determined by computational experiments and temperature measurement for optimal experiments.

The shape of the induction coil influences the temperature distribution in the specimen. Induction heating efficiency is dependent on the coil density and the distance between the coil and specimen surface. A concentrated coil density and a short distance to the specimen surface results in high heat flux. The coil has to be optimized for an optimal smooth axial temperature distribution in both heating and cooling processes. Temperature is measured with a thermocouple (type K) installed with compression junctions to the specimen. The final axial temperature deviations within the gauge length 12mm are below $\pm5^\circ$C by total $650^\circ$C and within the gauge length 25mm are less than $\pm14^\circ$C, as shown in Fig. \ref{Fig:Temp-Distr}.


\begin{figure}[!htp]
\includegraphics[width=8.5cm]{plot_thermal_stability.pdf}
\includegraphics[width=8.5cm]{plot_thermal_stability_deviation.pdf}
\caption{Temperature distribution in thermomechanical fatigue test. (a) Variations of temperature at the upper and lower border of the measure region. (b) Deviations of the upper and lower temperature to the controlling temperature.}
\label{Fig:Temp-Distr}
\end{figure}

% \begin{figure}[!htp]
% \includegraphics[width=8.5cm]{plot_schematic_thermal_strain_IP.pdf}
% \includegraphics[width=8.5cm]{plot_schematic_thermal_strain_OP.pdf}
% \caption{Temperature distribution in thermomechanical fatigue test. (a) Variations of temperature at the upper and lower border of the measure region. (b) Deviations of the upper and lower temperature to the controlling temperature.}
% \label{Fig:plot_schematic_thermal_strain}
% \end{figure}

Experiments were conducted on an MTS tension-torsion closed-loop servo hydraulic testing machine. Cooling is achieved by enforced air convection and heating is achieved by the induction heating device. The system is capable of applying both axial and biaxial loads with the temperature variation simultaneously. Thus, uniaxial or biaxial stress state can be generated under a given certain temperature range in the specimen. Also the test requirements from ASTM standard E2368 \cite{ASTM2014a} should be considered. Fig. \ref{Fig:Equipments} shows the induction heating system. The strains were measured using the high temperature axial and biaxial extensometers which contact the specimen surface with two ceramic rods. The gauge length of the axial-torsional extensometers is 25mm.

\begin{figure}[!htp]
\centering{\includegraphics[width=8.5cm]{Equipments.pdf}}
\caption{Induction heating system with biaxial test system MTS809.}
\label{Fig:Equipments}
\end{figure}

\subsection{Fatigue tests}
According to the standard ASTM E2368 \cite{ASTM2014a}, a thermomechanical fatigue cycle requests uniform temperature and strain fields over the specimen gage section which vary simultaneously and independently under given loading conditions. In the present work several factors may influence experimental results and the definitions are given as follows:

\begin{itemize}
  \item {\em Thermal strain}, $\varepsilon_{th}$, is induced by temperature and assumed to be linear proportional to temperature increment, $\varepsilon_{th}=\alpha \Delta T$. The thermal strain is generally isotropic.
  \item {\em Mechanical strain}, $\varepsilon_{m}$, is resulting from applied load and directly related with material deformation and damage. Under tension-torsion loading conditions, the mechanical deformation in testing is characterized by the maximum principal strain.
  \item {\em Total strain}, $\varepsilon_t$, is measured by the extensometer and denotes the sum of the thermal and mechanical strains, $\varepsilon_t=\varepsilon_m+\varepsilon_{th}$. Since the fatigue loading amplitudes are generally small, strains are additive.
  \item {\em Loading ratio}, $R_{\varepsilon}=\varepsilon_{m,\min}/\varepsilon_{m,\max}$, denotes the ratio of the minimum mechanical strain over the maximum mechanical strain within a loading cycle. Under axial-torsional loading condition, the mechanical strains are the principal strains.
  \item {\em Phase angle of the thermal loading and mechanical loading}, $\theta_{T-\varepsilon}$, represents the time difference between maximum values of temperature and mechanical strain. %For multi-axial cases the phase angle is oriented to the principal strain.
  \item {\em Phase angle of the thermal loading and torsinal loading}, $\theta_{T-\gamma}$, represents the time difference between maximum values of temperature and shear strain.
  \item {\em Phase angle of the mechanical loading and torsinal loading}, $\theta_{\varepsilon-\gamma}$, represents the time difference between maximum values of mechanical strain and shear strain.
  \item {\em Equivalent mechanical strain}, $\varepsilon_{eq}=\sqrt {\varepsilon _m^2 + {\gamma ^2}/3}$, is calculated by the von Mises relation\cite{Pol1991Cyclic}.
\end{itemize}

\begin{figure}
  \begin{minipage}[t]{0.5\linewidth}
  \nonumber
    \centering
    \includegraphics[width=4.5cm]{load_path_1.pdf}
    \centerline{\small (a) Tension-Compression}
    \centerline{\small (TC).}
  \end{minipage}%
  \begin{minipage}[t]{0.5\linewidth}
    \centering
    \includegraphics[width=4.5cm]{load_path_2.pdf}
    \centerline{\small (b) Proportional Path}
    \centerline{\small (PRO).}
  \end{minipage}
  \centering
  \begin{minipage}[t]{0.5\linewidth}
  \nonumber
    \centering
    \includegraphics[width=4.5cm]{load_path_3.pdf}
    \centerline{\small (c) Diamond Path (NPR).}
  \end{minipage}%
  % \begin{minipage}[t]{0.5\linewidth}
  %   \centering
  %   \includegraphics[width=4.5cm]{load_path_4.pdf}
  %   \centerline{\small (d) Cross Path.}
  % \end{minipage}
  \caption{Strain paths in thermo-mechanical fatigue tests.}
  \label{Fig:LoadPath}
\end{figure}

\begin{figure}[!htp]
\centering{\includegraphics[width=8.5cm]{plot_elastic_by_temperature_in718.pdf}}
\caption{Elastic material property as a function of temperature. \marked{The vertical axis has to be changed.}}
\label{Fig:plot_elastic_by_temperature_in718}
\end{figure}

All fatigue tests were performed in the temperature interval from 300$^\circ$C to 650$^\circ$C under mechanical strain control with fully reversed mechanical strain loading $R_{\varepsilon}=-1$. The experimental conditions of multi-axial thermomechanical fatigue tests are summarized in Table \ref{Tab:TestMatrix} with respect to testing parameters such as test type, mechanical and torsional strain amplitudes, equivalent strain amplitude and rates as well as phase shifts between mechanical strain, shear strain and temperature. All tests were carried out in air. Fig. \ref{Fig:LoadPath} shows schematically three strain paths for uniaxial tension/compression (Fig. \ref{Fig:LoadPath}(a)), proportional (Fig. \ref{Fig:LoadPath}(b)), and non-proportional (Fig. \ref{Fig:LoadPath}(c)) tests. Fig. \ref{Fig:LoadPath}(b) represents a 45$^\circ$ proportional loading with the ratio $\varepsilon_{m}/(\gamma/\sqrt{3})=1$. While Fig. \ref{Fig:LoadPath}(c) shows a diamond non-proportional strain path (i.e. the ratio of mechanical strain amplitude and shear strain amplitude is 1). Loading ratio $R_\varepsilon$ means the ratio of minimum mechanical strain to the maximum mechanical strain in a loading cycle. Under the axial-torsional condition, the loading ratio is calculated from the principal strains. It should be noted that the thin-walled tubular specimen is used in this study. 
In the tubular specimen, the shear strain and shear stress are not constant in the cross-section, and the shear stress and shear strain are always the greatest at the outer diameter.
Under elasto-plastic loading conditions, shear stress varies nonlinearly through the thin wall of the tubular specimen.
Therefore, according to ASTM E2207 \cite{ASTM2014}, the assumption of a uniformly distributed shear stress is recommended.
The shear stresses reported in this study are calculated dependent on the assumption as:
\begin{equation}
\tau=\frac{16T}{\pi(d_o^2-d_i^2)(d_o+d_i)},
\end{equation}
where, $T$ is the torsional moment, $\tau$ is the shear stress, $d_o$ and $d_i$ are the outer and inner diameters of the thin-walled hollow specimen, respectively. 
The effect of temperature variation on shear strain measurement can be neglected, according to Ref. \cite{Bakis2014}. Consequently, the compensation of torsional thermal strain is not performed in the present work. The shear strains are obtained directly from the measurement results by the extensometer.
%$N_f$ was defined as number of cycles at failure, i.e. $N_f$ represents the last cycle.$N_f$

%The multi-axial fatigue tests ran under given axial strain $\varepsilon$ and shear strain $\gamma$ in a cycle. The proportional and the non-proportional loading paths are shown in Figures \ref{Fig:LoadPath}. While Fig. \ref{Fig:LoadPath} (b) shows a proportional load,  Figs. \ref{Fig:LoadPath} (c) and (d) define two non-proportional loads, with very different characteristics. Loading ratio $R_\varepsilon$ means the ratio of minimum mechanical strain to the maximum mechanical strain in a loading cycle. Under the tension-torsion condition, the loading ratio is calculated from the principal strains.

Temperature was measured with three chromel-alumel (type K) thermocouples wrapped around the specimen. 

In comparison with isothermal fatigue test, the thermomechanical fatigue test is carried out under varying temperature, as shown in Fig. \ref{Fig:Temp-Distr}, in which the thermomechanical loads are characterized additionally by the phase angle, $\varphi$. In the present work, the temperature cycled from $T_{min}=300^\circ$C to $T_{max}=650^\circ$C at a heating/cooling rates of 3.89K/s with the constant ramp rates forming a triangular cycle. The test matrix for the present work is summarized in Table \ref{Tab:TestMatrix}, with both varying loading amplitudes and different phase angles. Effects of the loading proportionality is an additional issue to be clarified experimentally.

The mechanical strain component ($\varepsilon_{m}$) resulting when the free expansion thermal strain component ($\varepsilon_{th}$) is subtracted from the total strain ($\varepsilon_{t}$). In order to account for the thermal expansion during the tests, each specimen was heated by the induction heating device to the intended thermal cycle of the experiment under the  stress free condition before TMF tests. Under this condition the specimen can expand freely and the thermal expansion was measured directly.
Figure \ref{Fig:plot_elastic_by_temperature_in718} shows variations of elasticity modulus, shear modulus and Poisson's ratio as functions of temperature. The curves are independent of mechanical loading.


\begin{table*}[htbp]
  \centering
  \caption{Test matrix} \vspace{0.1cm}
    \begin{tabular}{p{2cm}p{1.5cm}p{1.5cm}p{1.5cm}p{2.5cm}p{1cm}p{1cm}p{1cm}p{1cm}}
    \hline
    Test Type & $\pm \varepsilon _m$ & $\pm \gamma/ \sqrt 3$ & $\varepsilon _{eq}$ & $\dot \varepsilon _{eq}$ & $\theta_{T-\varepsilon}$ & $\theta_{T-\gamma}$ & $\theta_{\varepsilon-\gamma}$ & $N_f$ \\
          & [\%]  & [\%]  & [\%]  & [s$^{-1}$] & [$^\circ$] & [$^\circ$] & [$^\circ$] &  \\
    \hline
    TC-IF & 1.00  & -     & 1.00  & $1\times 10^{-3}$ & -     & -     & -     & 231 \\
          & 0.80  & -     & 0.80  & $1\times 10^{-3}$ & -     & -     & -     & 326 \\
          & 0.70  & -     & 0.70  & $1\times 10^{-3}$ & -     & -     & -     & 592 \\
          & 0.60  & -     & 0.60  & $1\times 10^{-3}$ & -     & -     & -     & 1336 \\
          & 0.50  & -     & 0.50  & $1\times 10^{-3}$ & -     & -     & -     & 8449 \\
          & 0.45  & -     & 0.45  & $1\times 10^{-3}$ & -     & -     & -     & 15497 \\
          & 0.40  & -     & 0.40  & $6.4\times 10^{-3}$ & -     & -     & -     & 130585 \\
    \hline
    TC-IP & 1.00  & -     & 1.00  & $2.22\times 10^{-4}$ & 0     & -     & -     & 58 \\
          & 0.80  & -     & 0.80  & $1.78\times 10^{-4}$ & 0     & -     & -     & 176 \\
          & 0.70  & -     & 0.70  & $1.56\times 10^{-4}$ & 0     & -     & -     & 248 \\
          & 0.60  & -     & 0.60  & $1.33\times 10^{-4}$ & 0     & -     & -     & 1297 \\
    \hline
    TC-OP & 1.00  & -     & 1.00  & $2.22\times 10^{-4}$ & 180   & -     & -     & 209 \\
          & 0.80  & -     & 0.80  & $1.78\times 10^{-4}$ & 180   & -     & -     & 303 \\
          & 0.70  & -     & 0.70  & $1.56\times 10^{-4}$ & 180   & -     & -     & 429 \\
          & 0.65  & -     & 0.65  & $1.44\times 10^{-4}$ & 180   & -     & -     & 633 \\
    \hline
    TC-90 & 1.00  & -     & 1.00  & $2.22\times 10^{-4}$ & 90    & -     & -     & 387 \\
    \hline
    PRO-IP & 0.71  & 0.71  & 1.00  & $2.22\times 10^{-4}$ & 0     & 0     & 0     & 260 \\
          & 0.57  & 0.57  & 0.80  & $1.78\times 10^{-4}$ & 0     & 0     & 0     & 288 \\
          & 0.57  & 0.57  & 0.80  & $1.78\times 10^{-4}$ & 0     & 0     & 0     & 550 \\
          & 0.42  & 0.42  & 0.60  & $1.33\times 10^{-4}$ & 0     & 0     & 0     & 2848 \\
    \hline
    NPR-IP & 1.00  & 1.00  & 1.00  & $2.22\times 10^{-4}$ & 0     & 90    & 90    & 43 \\
          & 0.80  & 0.80  & 0.80  & $1.78\times 10^{-4}$ & 0     & 90    & 90    & 45 \\
          & 0.70  & 0.70  & 0.70  & $1.56\times 10^{-4}$ & 0     & 90    & 90    & 54 \\
          & 0.70  & 0.70  & 0.70  & $1.56\times 10^{-4}$ & 0     & 90    & 90    & 220 \\
          & 0.60  & 0.60  & 0.60  & $1.33\times 10^{-4}$ & 0     & 90    & 90    & 152 \\
          & 0.50  & 0.50  & 0.50  & $1.11\times 10^{-4}$ & 0     & 90    & 90    & 2544 \\
    \hline
    \end{tabular}%
  \label{Tab:TestMatrix}%
\end{table*}%


\marked{Additional references:
RC Rice, JL Jackson, J Bakuckas, S Thompson.
Metallic Materials Properties
Development and Standardization
(MMPDS). DOT/FAA/AR-MMPDS-01, Office of Aviation Research
Washington, D.C. 20591. 2003.
\newline
WD Klopp. Aerospace Structural Metals Handbook. More details have to be added.}




\section{ Test Results}
Experiments are started with uniaxial cyclic stress-strain curves to identify primary material property, as shown in Table \ref{tab:MechanicalProperties}. In the table experimental results by Kim \cite{kim1988elevated} and Nelson  \cite{nelson1992creep} as well as Mohabia \cite{Mahobia2014}  are included for comparison.
Heat treatments and processing of all the specimens are similar to those of the present work described in the previous section.

\begin{table*}[htbp]
  \centering
  \caption{Basic properties of Nickel-based superalloy Inconel 718 at 650$^{\circ}$C.}
    \begin{tabular}{llllllll}
    \hline
          & $E$     & $K'$     & $n'$     & $\sigma_f$    & $b$     & $\varepsilon_f$    & $c$ \\
    \hline
    NASA \cite{kim1988elevated, nelson1992creep}  & 162.6 GPa & 1827 MPa  & 0.16723 & 1348 MPa & -0.10052 & 0.12445 & -0.55218 \\
    THU   & 167.1 GPa & 1406 MPa  & 0.10527 & 1034 MPa & -0.04486 & 0.11499 & -0.52436 \\
    BHU \cite{Mahobia2014}   & 177.2 GPa & 1420 MPa  & 0.11332 & 985 MPa & -0.03917 & 0.24721 & -0.55682 \\
    \hline
    \end{tabular}%
  \label{tab:MechanicalProperties}%
\end{table*}%

In the table one sees partially significant difference among three series of testing, which may be induced by different material providers.  From this observation one may expect obvious deviations in fatigue performance for the nominal same superalloy Inconel 718.

For simplicity, the stress-strain relationship for both monotonic and cyclic loadings is described by the known Ramberg-Osgood model, as shown in Fig. \ref{Fig:plot_monotonic_cyclic_osgood}. The figure confirms that strain hardening can be expressed in the power-law function fairly well. The material demonstrates significant cyclic softening, of which the monotonic stress-strain curve can be expressed by the conventional Ramberg-Osgood model, as
\begin{equation}
{\varepsilon } = \frac{{\sigma }}{{E}} + {\left( {\frac{{\sigma }}{{K}}} \right)^{1/n}}
\end{equation}
for the monotonic loading, where $K$ is the material plastic offset and $n$ is the monotonic strain hardening exponent.
The cyclic stress-strain relationship is assumed to be in the form of the Ramberg-Osgood model, as:
\begin{equation}
\frac{{\Delta \varepsilon }}{2} = \frac{{\Delta \sigma }}{{2E}} + {\left( {\frac{{\Delta \sigma }}{{2K'}}} \right)^{1/n'}},
\end{equation}
where $K'$ is the cyclic strength coefficient and $n'$ is the cyclic strain hardening exponent.
Comparison between the present experiments and Ramberg-Osgood model is illustrated in Fig. \ref{Fig:plot_exp_half_life_cycle}. The experimental loops are taken from $N_f/2$th. cycles.
More detailed discussions about cyclic plasticity for the Inconel718 are referred a separate publication by the authors \cite{Sun2017}.

\begin{figure}[!htp]
\centering{\includegraphics[width=8.5cm]{plot_monotonic_cyclic_osgood.pdf}}
\caption{Monotonic and cyclic stress-strain curve with Ramberg-Osgood relationship for Inconel 718. \marked{(Add the monotonic experimental data. They are necessary.)}}
\label{Fig:plot_monotonic_cyclic_osgood}
\end{figure}


\begin{figure}[!htp]
\centering{\includegraphics[width=8.5cm]{plot_exp_half_life_cycle-TC-IF.pdf}}
\caption{Comparison of the stress-strain hysteresis loops of the half life cycle $N_f/2$ between experiments and Ramberg-Osgood model, with strain ranges of $\pm0.4\%, \pm0.5\%, \pm0.6\%, \pm0.7\%, \pm0.8\%$ and $\pm1.0\%$. \marked{(Reduce results. Just show  $\pm0.4\%,  \pm0.7\%$ and $\pm1.0\%$ results, but with comparison between experiments and Ramberg-Osgood model.)}}
\label{Fig:plot_exp_half_life_cycle}
\end{figure}


The Manson-Coffin model is popular in engineering fatigue life assessment, in which the applied strain amplitude $\Delta \varepsilon/2$ is decomposed in elastic strain amplitude ($\Delta \varepsilon_e/2$) and plastic strain amplitude ($\Delta \varepsilon_p/2$). The fatigue life, in terms of number of reversals to failure ($2N_f$), is determined from
\begin{equation}
\frac{{\Delta \varepsilon }}{2} = \frac{{\Delta {\varepsilon _e}}}{2} + \frac{{\Delta {\varepsilon _p}}}{2} = \frac{{{{\sigma '}_f}}}{E}{\left( {2{N_f}} \right)^b} + {\varepsilon '_f}{\left( {2{N_f}} \right)^c},
\label{Equ:CoffinManson}
\end{equation}
where ${{{\sigma '}_f}}$ is the fatigue strength coefficient, $b$ is the fatigue strength exponent, ${{{\varepsilon '}_f}}$ is the fatigue ductility coefficient and $c$ is the fatigue ductility exponent. The model parameters can be determined from uniaxial fatigue data in combining with the Ramberg-Osgood model. The results are summarized in Table \ref{tab:MechanicalProperties} for all three series of experiments.

\begin{figure}[!htp]
\centering{\includegraphics[width=8.5cm]{plot_exp_coffin_manson.pdf}}
\caption{Manson-Coffin plots of the isothermal fatigue tests.}
\label{Fig:Baseline}
\end{figure}

\begin{figure}[!htp]
\centering{\includegraphics[width=8.5cm]{plot_exp_fatigue_life_tmf.pdf}}
\caption{Manson-Coffin plots of the TMF fatigue tests.}
\label{Fig:plot_exp_fatigue_life}
\end{figure}

Fatigue tests with three different loading paths should give a systematical overview about the thermomechanical behavior of the material. Under strain-controlled uniaxial cyclic loading with mean mechanical strain, both IP and OP TMF results show the cyclic softening behavior as well as the hysteresis loops reaches a stable state with increasing number of cycles. The difference between IP and OP TMF stress-strain variations was observed that a mean stress decreasing during the IP TMF tests but increasing during the OP TMF tests. The proportional path IP TMF test shows the similar cyclic softening behavior of the uniaxial IP TMF tests. Also a mean stress decreasing was investigated in both axial and hoop directions.

Figure \ref{Fig:plot_exp_half_life_cycle} shows the experimental results of the multiaxial tests with the diamond mechanical strain path and IP temperature cycles, strain amplitude $\varepsilon_{eq} = \pm1$\%. Inconel 718 shows the cyclic softening behavior at both 300$^\circ$C and 650$^\circ$C. And the softening behavior is observed much more at higher temperature than lower temperature.
Fig. \ref{Fig:plot_monotonic_cyclic_osgood} shows the Mises equivalent strain amplitude vs. number of cycles to failure under four loading conditions: uniaxial in phase IP TMF, uniaxial out of phase OP TMF, proportional axial-torsional IP TMF and non-proportional axial torsional (diamond path) IP TMF.

Firstly, we consider the in phase IP and out of phase OP effect under uniaxial tests. The out of phase OP cycles resulted in lives that are significantly longer than the in phase IP tests at the equivalent strain amplitude 1\%. As the strain amplitude decrease, the deviations of fatigue life Nf  becomes smaller.
Secondly, we consider the multiaxial effect on the fatigue life. Because the temperature cycles of all the IF TMF tests from 300$^\circ$C to 650$^\circ$C remain the same. It was observed previously that the fatigue life of the multiaxial diamond path in phase IP TMF tests is shorter than the uniaxial in phase IP TMF tests. But the fatigue life of the proportional load path IP TMF tests is longer than the uniaxial IP TMF tests. This phenomenon was suggested to be related to the number of equivalent plastic strain of the two different strain paths.

\section{Microstructural investigations}
\begin{figure*}
  \begin{minipage}[t]{0.5\linewidth} % 如果一行放2个图,用0.5,如果3个图,用0.33\
  \nonumber
    \centering
    \begin{overpic}[width=6.0cm]{7112-1.jpg}
      \put(0,65){\fcolorbox{white}{white}{(a)}}
      \put(50,40){\color{white}\thicklines\vector(1,1){15.5}}
    \end{overpic}
  \end{minipage}%
  \begin{minipage}[t]{0.5\linewidth}
    \centering
    \begin{overpic}[width=6.0cm]{7047-1.jpg}
      \put(0,65){\fcolorbox{white}{white}{(b)}}
      \put(50,40){\color{white}\thicklines\vector(3,1){25}}
    \end{overpic}
  \end{minipage}

  \begin{minipage}[t]{0.5\linewidth} % 如果一行放2个图,用0.5,如果3个图,用0.33\
  \nonumber
    \centering
    \begin{overpic}[width=6.0cm]{7033-1.jpg}
      \put(0,65){\fcolorbox{white}{white}{(c)}}
      \put(45,40){\color{white}\thicklines\vector(1,0){18}}
    \end{overpic}
  \end{minipage}%
  \begin{minipage}[t]{0.5\linewidth}
    \centering
    \begin{overpic}[width=6.0cm]{7040-3.jpg}
      \put(0,65){\fcolorbox{white}{white}{(d)}}
      \put(50,30){\color{white}\thicklines\vector(1,1){20}}
    \end{overpic}
  \end{minipage}

  \begin{minipage}[t]{0.5\linewidth} % 如果一行放2个图,用0.5,如果3个图,用0.33\
  \nonumber
    \centering
    \begin{overpic}[width=6.0cm]{7046-6.jpg}
      \put(0,65){\fcolorbox{white}{white}{(e)}}
      \put(60,40){\color{white}\thicklines\vector(-1,-2){11}}
    \end{overpic}
  \end{minipage}%

  \caption{Locations of crack initiation: (a)TC-IF 0.45\%, (b)TC-IP 0.6\%, (c)TC-OP 0.65\%, (d)RPO-IP 0.6\%, (e)NPR-IP 0.7\%, (f)TGMF-OP 0.45\%.}
  \label{Fig:crack_initiation}
\end{figure*}

\begin{figure*}
  \begin{minipage}[t]{0.5\linewidth} % 如果一行放2个图,用0.5,如果3个图,用0.33\
  \nonumber
    \centering
    \begin{overpic}[width=6.0cm]{7112-4.jpg}
      \put(0,65){\fcolorbox{white}{white}{(a)}}
      \put(50,50){\color{white}\thicklines\vector(-1,-1){20}}
    \end{overpic}
  \end{minipage}%
  \begin{minipage}[t]{0.5\linewidth}
    \centering
    \begin{overpic}[width=6.0cm]{7047-7.jpg}
      \put(0,65){\fcolorbox{white}{white}{(b)}}
      \put(50,50){\color{white}\thicklines\vector(-1,-1){20}}
    \end{overpic}
  \end{minipage}

  \begin{minipage}[t]{0.5\linewidth} % 如果一行放2个图,用0.5,如果3个图,用0.33\
  \nonumber
    \centering
    \begin{overpic}[width=6.0cm]{7033-16.jpg}
      \put(0,65){\fcolorbox{white}{white}{(c)}}
      \put(50,50){\color{white}\thicklines\vector(-1,1){20}}
    \end{overpic}
  \end{minipage}%
  \begin{minipage}[t]{0.5\linewidth}
    \centering
    \begin{overpic}[width=6.0cm]{7040-5.jpg}
      \put(0,65){\fcolorbox{white}{white}{(d)}}
      \put(50,30){\color{white}\thicklines\vector(-1,-1){20}}
    \end{overpic}
  \end{minipage}

  \begin{minipage}[t]{0.5\linewidth} % 如果一行放2个图,用0.5,如果3个图,用0.33\
  \nonumber
    \centering
    \begin{overpic}[width=6.0cm]{7046-8.jpg}
      \put(0,65){\fcolorbox{white}{white}{(e)}}
      \put(30,30){\color{white}\thicklines\vector(1,2){11}}
    \end{overpic}
  \end{minipage}%

  \caption{Observation of fatigue striations on fractures surface: (a)TC-IF 0.45\%, (b)TC-IP 0.6\%, (c)TC-OP 0.65\%, (d)PRO-IP 0.6\%, (e)NPR-IP 0.7\%}
  \label{Fig:fatigue_striations}
\end{figure*}

\section{ Thermomechanical Fatigue Assessment}

\subsection{Brown-Miller Model}
Based on the critical plane concepts \cite{Brown2006}, Wang and Brown \cite{Wang1993} proposed that the Kandil, Brown and Miller fatigue parameter \cite{Kandil1982} can be reformulated as the equivalent shear strain amplitude:
\begin{equation}
\frac{{\Delta \hat \gamma }}{2} = \frac{{\Delta {\gamma _{max}}}}{2} + S\Delta {\varepsilon _n},
\label{Equ:ShearStrainBM}
\end{equation}
where $\frac{{\Delta \hat \gamma }}{2}$ is the equivalent shear strain range, $\Delta {\varepsilon _n}$ represents the normal strain excursion on the on the plane with the maximum strain range $\Delta {\gamma _{max}}$. The material dependent parameter S represents the influence of the normal strain on the crack propagation.
The fatigue endurance is suggested as:
\begin{equation}
\frac{{\Delta \hat \gamma }}{2} = A\frac{{{{\sigma '}_f}}}{E}{\left( {2{N_f}} \right)^b} + B{{\varepsilon '}_f}{\left( {2{N_f}} \right)^c},
\end{equation}
with
\[A = 1 + {\nu _e} + \left( {1 - {\nu _e}} \right)S,\]
and
\[B = 1 + {\nu _p} + \left( {1 - {\nu _p}} \right)S.\]
\begin{figure}[!htp]
\centering{\includegraphics[width=8.5cm]{NF-NP-TMF-BM.pdf}}
\caption{Brown-Miller Model.}
\label{Fig:NF-NP-TMF-BM}
\end{figure}

\subsection{Fatemi-Socie Model}
Based on the work of Brown and Miller, Fatemi and Socie \cite{Fatemi1988} proposed that the normal strain term in Equation (\ref{Equ:ShearStrainBM}) should be replaced by the normal stress.
The equivalent shear strain amplitude is developed as:
%\begin{equation}
%\end{equation}
\begin{equation}
\frac{{\Delta \hat \gamma }}{2} = \frac{{\Delta {\gamma _{\max }}}}{2}\left( {1 + k\frac{{{\sigma _{n,max}}}}{{{\sigma _y}}}} \right),
\end{equation}
where
$\sigma _{n,max}$ is the maximum normal stress on the critical plane suffering the maximum strain range $\Delta {\gamma _{max}}$, $k$ is a material parameter, the sensitivity of the material to normal stress is reflected in the ratio $k/\sigma_y$.
They developed a damaging model oriented on the shear-based damage initiation:
\begin{equation}
\frac{{\Delta \hat \gamma }}{2} = \frac{{{{\tau '}_f}}}{G}{\left( {2{N_f}} \right)^{{b_0}}} + {{\gamma '}_f}{\left( {2{N_f}} \right)^{{c_0}}}.
\end{equation}
Furthermore, McClaflin and Fatemi \cite{McClaflin2004} proposed that the sensitivity parameter $k$ is varied with fatigue life and can be expressed by tension and torsion data, as
\begin{equation}
k =  \frac{{k_0 {\sigma _y}}}{{{{\sigma '}_f}{{\left( {2{N_f}} \right)}^b}}}
\end{equation}
with
\[
k_0 =  {\frac{{\frac{{{{\tau '}_f}}}{G}{{\left( {2{N_f}} \right)}^{{b_0}}} + {{\gamma '}_f}{{\left( {2{N_f}} \right)}^{{c_0}}}}}{{\left( {1 + {\nu _e}} \right)\frac{{{{\sigma '}_f}}}{E}{{\left( {2{N_f}} \right)}^b} + \left( {1 + {\nu _p}} \right){{\varepsilon '}_f}{{\left( {2{N_f}} \right)}^c}}} - 1} .
\]

%where the equivalent shear strain on the critical plane was based on a combination of the maximum shear strain amplitude $\Delta \gamma$ and maximum normal stress $\sigma _{n,max}$ during the cycle. The shear strain was used as the variable decisive for the rain-flow decomposition.
%The criterion used to be based on the MSSR maximization.
\begin{figure}[!htp]
\centering{\includegraphics[width=8.5cm]{NF-NP-TMF-FS.pdf}}
\caption{Fatemi-Socie Model.}
\label{Fig:NF-NP-TMF-FS}
\end{figure}

\subsection{Smith-Watson-Topper Model}
SWT parameter is based on the principal strain range Š€ŠÅ1 and the maximum stress on the plane of the principal
strain range

\[{\sigma _{n,max}}\frac{{\Delta \varepsilon }}{2} = \frac{{{{\sigma '}_f}^2}}{E}{\left( {2{N_f}} \right)^{2b}} + {\sigma '_f}{\varepsilon '_f}{\left( {2{N_f}} \right)^{b + c}}\]
\begin{figure}[!htp]
\centering{\includegraphics[width=8.5cm]{NF-NP-TMF-SWT.pdf}}
\caption{Smith-Watson-Topper Model.}
\label{Fig:NF-NP-TMF-SWT}
\end{figure}

\subsection{Liu's Strain Energy Models}
\begin{eqnarray*}
{\left( {\Delta {\sigma _n}\Delta {\varepsilon _n}} \right)_{\max }} + \left( {\Delta \tau \Delta \gamma } \right) &=& \frac{{4{{\sigma '}_f}^2}}{E}{\left( {2{N_f}} \right)^{2b}}
\\
& & + 4{{\sigma '}_f}{{\varepsilon '}_f}{\left( {2{N_f}} \right)^{b + c}}
\end{eqnarray*}

\begin{figure}[!htp]
\centering{\includegraphics[width=8.5cm]{NF-NP-TMF-Liu1.pdf}}
\caption{Liu Tension Strain Energy Model.}
\label{Fig:NF-NP-TMF-Liu}
\end{figure}

\begin{eqnarray*}
\left( {\Delta {\sigma _n}\Delta {\varepsilon _n}} \right) + {\left( {\Delta \tau \Delta \gamma } \right)_{\max }} &=& \frac{{4{{\tau '}_f}^2}}{G}{\left( {2{N_f}} \right)^{2b\gamma }}
\\
&& + 4{{\tau '}_f}{{\gamma '}_f}{\left( {2{N_f}} \right)^{b\gamma  + c\gamma }}
\end{eqnarray*}

\begin{figure}[!htp]
\centering{\includegraphics[width=8.5cm]{NF-NP-TMF-Liu2.pdf}}
\caption{Liu Shear Strain Energy Model.}
\label{Fig:NF-NP-TMF-Liu2}
\end{figure}

\subsection{Chu Strain Energy Model}
\begin{eqnarray*}
{\left( {{\tau _{n,\max }}\frac{{\Delta \gamma }}{2} + {\sigma _{n,\max }}\frac{{\Delta \varepsilon }}{2}} \right)_{\max }} &=& 1.02\frac{{{{\sigma '}_f}^2}}{E}{\left( {2{N_f}} \right)^{2b}} \\
&& + 1.04{{\sigma '}_f}{{\varepsilon '}_f}{\left( {2{N_f}} \right)^{b + c}}
\end{eqnarray*}

\begin{figure}[!htp]
\centering{\includegraphics[width=8.5cm]{NF-NP-TMF-Chu.pdf}}
\caption{Chu Strain Energy Model.}
\label{Fig:NF-NP-TMF-Chu}
\end{figure}



\marked{Figure \ref{Fig:plot_fatigue_life_quantitative_evaluation_tmf} shows a quantitative assessment of fatigue life prediction results. The comparison
between calculated Ncal and experimental Nexp fatigue lives has been made by means of two statistical parameters.
The first of them is a mean dispersion of fatigue life}

\[{T_N} = {10^{\bar E}}\]

where $\bar E$is given by:

\[\bar E = \frac{1}{n}\sum\limits_{i = 1}^n {\log \left( {\frac{{{N_{\exp ,i}}}}{{{N_{cal,i}}}}} \right)} \]
where n is a number of the compared results. The second parameter is a life prediction mean-squared error

\[{T_{RMS}} = {10^{{E_{RMS}}}}\]
where $E_{RMS}$ is given by:
\[{E_{RMS}} = \sqrt {\frac{1}{n}\sum\limits_{i = 1}^n {{{\log }^2}\left( {\frac{{{N_{\exp ,i}}}}{{{N_{cal,i}}}}} \right)} } \]

\marked{
$T_N$ equals 1 when mean experimental and calculated fatigue lives are equal; more than 1 when the experimental life values are higher than the calculated ones; lower than 1 when the experimental life values are lower than the calculated ones. The $T_N$ quantity is insensitive to the dispersion of life. It can assume the same value for the results with low and high statistical dispersion. Quantity $T_{RMS}$ is a measure of statistical dispersion. It assumes the value equal 1 when the mean and the statistical dispersion of experimental and computational life are identical. It is higher than 1 in other cases. $T_{RMS}$ does not provide any information on whether the computational life is higher or lower than the experimental one.}

\begin{figure}[!htp]
\centering
\begin{overpic}[width=8.5cm]{plot_fatigue_life_quantitative_evaluation_tmf_TN.pdf}
\put(14,65){\fcolorbox{white}{white}{(a)}}
\end{overpic}
\begin{overpic}[width=8.5cm]{plot_fatigue_life_quantitative_evaluation_tmf_TRMS.pdf}
\put(14,65){\fcolorbox{white}{white}{(b)}}
\end{overpic}
\caption{The quantitative evaluation of the fatigue life prediction results: (a)$T_N$, (b)$T_{RMS}$.}
\label{Fig:plot_fatigue_life_quantitative_evaluation_tmf}
\end{figure}

\section{Conlusions}
This paper has described the experimental facility of a non-proportional thermomechanical fatigue test. The machine had to be carefully calibrated from its original design to improve the thermal gradient and ensure the control stability of the axial-torsional extensometers.
A summary of the important mechanical test results is:
The uniaxial out of phase OP TMF test has a longer lifetime than the uniaxial in phase IP test within the Mises equivalent mechanical strain amplitude range from 0.5\% to 1\%.
Fatigue lives are strongly influenced by the strain path. At the same equivalent mechanical strain and the same in phase IP temperature conditions, the non-proportional diamond path test has a shorter lifetime than the uniaxial and 45? proportional path test.

The life predictions from the known fatigue models are unsatisfactory. The plots contains generally large scatterings, which implies improper fatigue variables in the models.
A TMF life prediction model has to consider influences from varying temperature as well as the non-proportional loads. The TMF life prediction is still an open issue.






\section*{Acknowledgement:} The present work is financed by the China Natural Science Foundation under the contract number 51175041.


\bibliographystyle{unsrt}            % bibliography style
%\bibliographystyle{plain}            % bibliography style
\bibliography{bibliography}          % personal bibliography file

\end{document} 