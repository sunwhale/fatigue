\documentclass{article}

\usepackage{ctex}
\newcommand{\marked}[1]{\textcolor{red}{#1}}

\begin{document}
\title{Life Assessment of Multi-Axial Thermomechanical Fatigue of A Nickel-Based Superalloy}

\author{袁荒,孙经雨}
\date{\today}
\maketitle

\section*{摘要}

\marked{Background/Introduction:}

Inconel 718镍基高温合金广泛用于燃气轮机热端部件,绝大多数部件在服役条件下都承受非比例热机械疲劳(TMF)载荷。

\marked{Subject of the paper:}

现有研究主要基于单轴TMF试验的结果,未考虑多轴载荷带来的影响,本文重点研究材料在比例和非比例加载下的TMF寿命预测。

\marked{Techniques and methods used:}

通过开展比例与非比例加载条件下的热机械疲劳试验,测试材料细观结构形貌、应力应变特性、疲劳寿命性能等,研究非比例热机械疲劳的失效机理。

\marked{Main Results:}

结果表明,在300-650$^{\circ}$C热机械疲劳条件下,没有发现明显的蠕变现象,低周疲劳和氧化为主要的失效模式。
在应变幅值较大的情况下,同相位(IP)热机械疲劳寿命要低于反相位(OP)热机械疲劳寿命。
对于相同的等效应变幅值,非比例加载条件下的TMF寿命要大大低于比例加载和单轴载荷的TMF寿命。
通过对疲劳断口形貌的分析,比例和非比例加载的TMF试验裂纹都萌生于试件的外表面,由于扭转载荷的存在,比例和非比例加载的TMF试验疲劳裂纹扩展区非常光滑,无法观测到明显的疲劳辉纹。

\marked{Conclusions:}

选取多种多轴疲劳模型对试验数据进行预测,误差较大,其中SWT模型预测的结果相对较好,在5倍分散带内。

\end{document} 